\documentclass{article}
\usepackage{latexsym}
\usepackage{graphicx}

\begin{document}
ID for the String `abaaaaba': (Here, the symbols in bold represent the
read--write head position.) (q0, abaaaaba) $\to $ (q1, Xbaaaaba) $\to $ (q1,
Xbaaaaba) $\to $ (q1, Xbaaaaba) $\to $ (q1, Xbaaaaba) $\to $ (q1, Xbaaaaba)
$\to $ (q1, Xbaaaaba) $\to $ (q1, Xbaaaaba) $\to $ (q1, XbaaaabaB) $\to $
(q2, XbaaaabaB) $\to $ (q3, XbaaaabX) $\to $ (q3, XbaaaabX) $\to $ (q3,
XbaaaabX) $\to $ (q3, XbaaaabX) $\to $ (q3, XbaaaabX) $\to $ (q3, XbaaaabX)
$\to $ (q3, XbaaaabXB) $\to $ (q3, XbaaaabX) $\to $ (q0, XbaaaabX) $\to $
(q4, XYaaaabX) $\to $ (q4, XYaaaabX) $\to $ (q4, XYaaaabX) $\to $ (q4,
XYaaaabX) $\to $ (q4, XYaaaabX) $\to $ (q4, XYaaaabX) $\to $ (q5, XYaaaabX)
$\to $ (q6, XYaaaaYX) $\to $ (q6, XYaaaaYX) $\to $ (q6, XYaaaaYX) $\to $
(q6, XYaaaaYX) $\to $ (q0, XYaaaaYX) $\to $ (q1, XYXaaaYX) $\to $ (q1,
XYXaaaYX) $\to $ (q1, XYXaaaYX) $\to $ (q1, XYXaaaYX) $\to $ (q2, XYXaaaYX)
$\to $ (q3, XYXaaXYX) $\to $ (q3, XYXaaXYX) $\to $ (q3, XYXaaXYX) $\to $
(q0, XYXaaXYX) $\to $ (q1, XYXXaXYX) $\to $ (q1, XYXXaXYX) $\to $ (q2,
XYXXaXYX) $\to $ (q3, XYXXXXYX) $\to $ (q0, XYXXXXYX) $\to $ (qf, XYXXXXYX)
[Halt]

Example 8.4 Design a TM to accept the language L $=$ \textbraceleft set of
all palindromes over a, b\textbraceright . Show the IDs for the null string,
`a', `aba', and `baab'.

Solution: Palindromes are of two types:

i) Odd palindromes, where the number of characters is odd

ii) Even palindrome, where the number of characters is even. A null string
is also a palindrome.

A string which starts with `a' or `b' must end with `a' or `b',
respectively, if the string is a palindrome. If a string starts with `a',
that `a' is replaced by a blank symbol `B' upon traversal with a state
change from q1 to q2 and the right shift of the read--write head. The
transitional function is

$\delta $(q1, a) $\to $ (q2, B, R)

Then, the machine needs to search for the end of the string. The string must
end with `a', and that `a' exists before a blank symbol B at the right hand
side. Before getting that blank symbol, the machine needs to traverse all
the remaining `a' and `b' of the string. The transitional functions are

$\delta $(q2, a) $\to $ (q2, a, R)

$\delta $(q2, b) $\to $ (q2, b, R)

$\delta $(q2, B) $\to $ (q3, B, L)

$\delta $(q3, a) $\to $ (q4, B, L)

The machine now needs to search for the second symbol (from the starting) of
the string. That symbol exists after the blank symbol B, which is replaced
at the ?rst. Before that the machine needs to traverse the remaining `a' and
`b' of the string. The transitional functions are

$\delta $ (q4, a) $\to $ (q4, a, L)

$\delta $ (q4, b) $\to $ (q4, b, L)

$\delta $ (q4, B) $\to $ (q1, B, R)

If the string starts with `b', the transitional functions are the same as
`a' but some states are changed. The transitional functions are

$\delta $ (q1, b) $\to $ (q5, B, R)

$\delta $ (q5, a) $\to $ (q5, a, R)

$\delta $ (q5, b) $\to $ (q5, b, R)

$\delta $ (q5, B) $\to $ (q6, B, L)

$\delta $ (q6, b) $\to $ (q4, B, L)

When all `a' and `b' are traversed and replaced by B, the states may be one
of q3 or q6, if the last symbol traversed is `a' or `b', respectively.
Transitional functions for acceptance are

$\delta $ (q3, B) $\to $ (q7, B, H)

$\delta $ (q6, B) $\to $ (q7, B, H)

A null string is also a palindrome. On the tape, a null symbol means blank
B. In state q1, the machine gets the symbol. The transitional function is

$\delta $ (q1, B) $\to $ (q7, B, H)

The transitional functions in tabular form are represented as follows.

\begin{figure}[htbp]
\centerline{\includegraphics[width=10.03in,height=3.47in]{page1.eps}}
\label{fig1}
\end{figure}

ID for Null String

\begin{center}
(q1, B) $\to $ (q7, B) [Halt]
\end{center}

ID for the String `a'

\begin{center}
(q1, aB) $\to $ (q2, BB) $\to $ (q3, BB) $\to $ (q7, BB) [Halt]
\end{center}

ID for the String `aba'

\begin{center}
(q1, abaB) $\to $ (q2, BbaB) $\to $ (q2, BbaB) $\to $ (q2, BbaB) $\to $ (q3,
BbaB) $\to $ (q4, BbBB) $\to $ (q4, BbBB) $\to $ (q1, BbBB) $\to $ (q1,
BBBB) $\to $ (q7, BBBB) [Halt]
\end{center}

ID for the String `baab'

\begin{center}
(q1, baabB) $\to $ (q5, BaabB) $\to $ (q5, BaabB) $\to $ (q5, BaabB) $\to $
(q5, BaabB) $\to $ (q6, BaabB) $\to $ (q4, BaaBB) $\to $ (q4, BaaBB) $\to $
(q4, BaaBB) $\to $ (q1, BaaBB) $\to $ (q2, BBaBB) $\to $ (q2, BBaBB) $\to $
(q3, BBaBB) $\to $ (q3, BBBBB) $\to $ (q7, BBBBB) [Halt].
\end{center}

Example 8.5 Design a TM to accept the language L $=$ anbncn, where n $\ge $
1.

Solution: This is a context-sensitive language. The language consists of a,
b, and c. Here, the number of `a' is equal to the number of `b' which is
equal to the number of `c'. n number `c' is followed by n number of `b'
which are followed by n number of `a'. The TM is designed as follows. For
each `a', search for `b' and `c' and again traverse back to the left to
search for the leftmost `a'. `a' is replaced by `X', `b' is replaced by `Y',
and `c' is replaced by `Z'. When all the `a' are traversed and replaced by
`X', the machine traverses the right side to ? nd if any untraversed `b' or
`c' is left or not. If not, the machine gets Y and Z and at last a blank.
Upon getting the blank symbol, the machine halts. The transitional functions
are

$\delta $ (q1, a) $\to $ (q2, X, R) // `a' is traversed

$\delta $ (q2, a) $\to $ (q2, a, R) // the remaining `a's are traversed

$\delta $ (q2, b) $\to $ (q3, Y, R) // `b' is traversed

$\delta $ (q3, b) $\to $ (q3, b, R) // the remaining `b's are traversed

$\delta $ (q3, c) $\to $ (q4, Z, L) // `c' is traversed

$\delta $ (q4, b) $\to $ (q4, b, L) // the remaining `b's are traversed from
right to left

$\delta $ (q4, Y) $\to $ (q4, Y, L) // Y, replacement for `b' is traversed

$\delta $ (q4, a) $\to $ (q4, a, L) // the remaining `a's are traversed from
right to left

$\delta $ (q4, X) $\to $ (q1, X, R) // rightmost X, replacement of `a' is
traversed

$\delta $ (q2, Y) $\to $ (q2, Y, R) // this is used from the second time
onwards

$\delta $ (q3, Z) $\to $ (q3, Z, R) // this is used from the second time
onwards

$\delta $ (q4, Z) $\to $ (q4, Z, L) // this is used from the second time
onwards

$\delta $ (q1, Y) $\to $ (q5, Y, R) // this is used when all the `a' are
traversed

$\delta $ (q5, Y) $\to $ (q5, Y, R) // this is used when all the `a' are
traversed and the machine is traversing right

$\delta $ (q5, Z) $\to $ (q6, Z, R) // this is used when all the `a' and `b'
are traversed.

$\delta $ (q6, Z) $\to $ (q6, Z, R) // this is used when all the `a' and `b
are traversed and the machine is traversing right to search for the
remaining `c'

$\delta $ (q6, B) $\to $ (qf , B, H) // this is used when all the `a', `b',
and `c' are traversed

Upon executing the last transitional function, the machine halts.

The transitional functions in tabular form are represented as follows.

TRIAL RESTRICTION

Example 8.6 Design a TM to perform the concatenation operation on string of
`1'. Show an ID for w1 $=$ 111 and w2 $=$ 1111.

Solution: The TM does the concatenation operation on two strings w1 and w2.
These two strings are placed on the tape of the TM separated by the blank
symbol B. After traversal, the blank symbol is removed and the two strings
are concatenated.

The transitional functions are

$\delta $ (q1, 1) $\to $ (q1, 1, R) // traversing `1' of the ? rst string

$\delta $ (q1, B) $\to $ (q2, 1, R) // traversing the separating blank symbol
and converting it to `1'

$\delta $ (q2, 1) $\to $ (q2, 1, R) // traversing `1' of the second string

$\delta $ (q2, B) $\to $ (q3, B, L) // blank symbol traversal means the end of
the second string

$\delta $ (q3, 1) $\to $ (q4, B, L) // the last`1' of the second string is
converted to `B' to keep the number of`1' the same

$\delta $ (q4, 1) $\to $ (q4, 1, L) // traversing the remaining `1' at the
left side

$\delta $ (q4, B) $\to $ (q5, B, H) // halts

\textit{[This machine is also applicable for performing Z }$= X +$\textit{ Y, where X }$=$\textit{ \textbar w1\textbar and Y }$=$\textit{ \textbar w2\textbar .] }

ID for w1 $=$ 111 and w2 $=$ 1111 and the result is 1111111

(q1, 111B1111B) $\to $ (q1, 111B1111B) $\to $ (q1, 111B1111) $\to $ (q1,
111B1111B) $\to $ (q2, 1111111B) $\to $

(q2, 11111111B) $\to $ (q2, 11111111B) $\to $ (q2, 11111111B) $\to $ (q2,
11111111B) $\to $ (q3, 11111111B) $\to $ (q4, 1111111BB) $\to $ (q4,
1111111BB) $\to $ (q4, 1111111BB) $\to $ (q4, 1111111BB) $\to $ (q4,
B1111111BB) $\to $ (q4, B1111111BB) $\to $ (q4, B1111111BB) $\to $ (q4,
B1111111BB) $\to $ (q5, B1111111BB) [Halt]

Example 8.7 Design a TM to perform the following operation

f(x, y) $=$ x $-$y, where x \textgreater y Show an ID for 4 $-$ 2 $=$ 2.

Solution: This performs the subtraction operation. Here, x and y both are
integer numbers. Let x $=$ \textbar W1\textbar and y $=$ \textbar W2\textbar
, where W1 and W2 are the strings of `1'. The two strings are placed on the
input tape separated by B.

Starting traversal from the left hand side `1', the machine moves towards
the right. It traverses B with a state changed from q0 to q1. In state q1,
it again traverses right to ? nd B, that is, the end of the second string.
It changes its state and traverses left and changes the rightmost `1' by B
(with state change and traversing left). It traverses the separating `B'
symbol and all `1's of the ? rst string and ? nds the beginning of the ? rst
string, which starts after B at the left hand side. It changes the ? rst `1'
by B and again traverses right. This process continues till all the `1's of
the second string are replaced by `B'. The transitional functions are

$\delta $ (q0, 1) $\to $ (q0, 1, R) // all the `1's of the ? rst string are
traversed

$\delta $ (q0, B) $\to $ (q1, B, R) // the separating B is traversed

$\delta $ (q1, 1) $\to $ (q1, 1, R) // all the `1's of the second string are
traversed

$\delta $ (q1, B) $\to $ (q2, B, L) // to signify the end of the second string

\end{document}
